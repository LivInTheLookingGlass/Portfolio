\documentclass[10pt, letterpaper]{article}

% Packages:
\usepackage[
	ignoreheadfoot, % set margins without considering header and footer
	top=1 cm, % seperation between body and page edge from the top
	bottom=1 cm, % seperation between body and page edge from the bottom
	left=1.25 cm, % seperation between body and page edge from the left
	right=1.25 cm, % seperation between body and page edge from the right
	footskip=0 cm, % seperation between body and footer
	% showframe % for debugging
]{geometry} % for adjusting page geometry
\usepackage{titlesec} % for customizing section titles
\usepackage{tabularx} % for making tables with fixed width columns
\usepackage{array} % tabularx requires this
\usepackage[dvipsnames]{xcolor} % for coloring text
\definecolor{primaryColor}{RGB}{0, 0, 0} % define primary color
\usepackage{enumitem} % for customizing lists
\usepackage{fontawesome5} % for using icons
\usepackage{amsmath} % for math
\usepackage[
	pdftitle={Olivia Appleton-Crocker's Resume},
	pdfauthor={Olivia Appleton-Crocker},
	pdfcreator={LaTeX with RenderCV},
	colorlinks=true,
	urlcolor=primaryColor
]{hyperref} % for links, metadata and bookmarks
\usepackage[pscoord]{eso-pic} % for floating text on the page
\usepackage{calc} % for calculating lengths
\usepackage{bookmark} % for bookmarks
\usepackage{lastpage} % for getting the total number of pages
\usepackage{changepage} % for one column entries (adjustwidth environment)
\usepackage{paracol} % for two and three column entries
\usepackage{ifthen} % for conditional statements
\usepackage{needspace} % for avoiding page brake right after the section title
\usepackage{iftex} % check if engine is pdflatex, xetex or luatex
\usepackage{xfp}

% Ensure that generate pdf is machine readable/ATS parsable:
\ifPDFTeX
\input{glyphtounicode}
\pdfgentounicode=1
\usepackage[T1]{fontenc}
\usepackage[utf8]{inputenc}
\usepackage{lmodern}
\fi

\usepackage{charter}

% Some settings:
\raggedright
\AtBeginEnvironment{adjustwidth}{\partopsep0pt} % remove space before adjustwidth environment
\pagestyle{empty} % no header or footer
\setcounter{secnumdepth}{0} % no section numbering
\setlength{\parindent}{0pt} % no indentation
\setlength{\topskip}{0pt} % no top skip
\setlength{\columnsep}{0.15cm} % set column seperation
\pagenumbering{gobble} % no page numbering

\titleformat{\section}{\needspace{4\baselineskip}\bfseries\large}{}{0pt}{}[
\titlerule]

\titlespacing{\section}{
% left space:
-1pt }{
% top space:
0.3 cm }{
% bottom space:
\subsectiongapd } % section title spacing

\renewcommand{\labelitemi}{$\vcenter{\hbox{\small$\bullet$}}$} % custom bullet points
\newenvironment{highlights}{ \begin{itemize}[ topsep=\subsectiongap, parsep=\subsectiongap, partopsep=0pt,
itemsep=0pt, leftmargin=0 cm + 10pt ] }{ \end{itemize} } % new environment for highlights

\newenvironment{highlightsforbulletentries}{ \begin{itemize}[ topsep=\subsectiongap,
parsep=\subsectiongap, partopsep=0pt, itemsep=0pt, leftmargin=10pt ] }{ \end{itemize} } % new environment for highlights for bullet entries

\newenvironment{onecolentry}{ \begin{adjustwidth}{ 0 cm + 0.00001 cm }{ 0 cm + 0.00001 cm }
}{ \end{adjustwidth} } % new environment for one column entries

\newenvironment{twocolentry}[2][]{ \onecolentry \def\secondColumn{#2} \setcolumnwidth{\fill, 4.5 cm}
\begin{paracol}{2} }{ \switchcolumn \raggedleft \secondColumn  \end{paracol}
\endonecolentry } % new environment for two column entries

\newenvironment{threecolentry}[3][]{ \onecolentry \def\thirdColumn{#3} \setcolumnwidth{, \fill, 4.5 cm}
\begin{paracol}{3} {\raggedright #2} \switchcolumn }{ \switchcolumn \raggedleft \thirdColumn
\end{paracol} \endonecolentry } % new environment for three column entries

\newenvironment{header}{
\setlength{\topsep}{0pt}
\par\kern\topsep
\centering
\linespread{1.5} }{ \par\kern\topsep } % new environment for the header

\newcommand{\placelastupdatedtext}{% \placetextbox{<horizontal pos>}{<vertical pos>}{<stuff>}
\AddToShipoutPictureFG*{% Add <stuff> to current page foreground
\put( \LenToUnit{\paperwidth-2 cm-0 cm+0.05cm}, \LenToUnit{\paperheight-1.0 cm} ){\vtop{{\null}\makebox[0pt][c]{ \small\color{gray}\textit{Last updated in January 2025}\hspace{\widthof{Last updated in January 2025}} }}}%
}%
}%

\newcommand{\sectionoffset}{-0.12 cm}
\newcommand{\ssubsectiongap}{ 0.0953}
\newcommand{\ssubsectiongapd}{\fpeval{2 * \ssubsectiongap}}
\newcommand{\subsectiongap}{\ssubsectiongap cm}
\newcommand{\subsectiongapd}{\ssubsectiongapd cm}

% save the original href command in a new command:
\let\hrefWithoutArrow\href

% new command for external links:

\begin{document}
	\newcommand{\AND}{\unskip \cleaders\copy\ANDbox\hskip\wd\ANDbox \ignorespaces }
	\newsavebox{\ANDbox}
	\sbox{\ANDbox}{$|$}

	\begin{header}
		\fontsize{25 pt}{25 pt}\selectfont Olivia Appleton-Crocker

		\normalsize \mbox{Chicago, IL}%
		\kern 3.0 pt%
		\AND%
		\kern 3.0 pt%
		\mbox{\hrefWithoutArrow{tel:+1-906-361-9876}{+\!1-906-361-9876}}%
		\kern 3.0 pt%
		\AND%
		\kern 3.0 pt%
		\mbox{\hrefWithoutArrow{https://oliviaappleton.com/}{oliviaappleton.com}}%
		\kern 3.0 pt%
		\AND%
		\kern 3.0 pt%
		\mbox{\hrefWithoutArrow{mailto:liv@oliviaappleton.com}{liv@oliviaappleton.com}}%
		\kern 3.0 pt%
		\AND%
		\kern 3.0 pt%
		\mbox{\hrefWithoutArrow{https://github.com/LivInTheLookingGlass}{github.com/LivInTheLookingGlass}}%
	\end{header}

	\vspace{-8 pt}

        \section{Summary}

        \begin{onecolentry}
        Software Engineer and Data Scientist with expertise in Python, software design, and systems programming. Experienced in research, GIS, and industry R\&D; solving complex challenges in distributed systems \& open-source development (including CPython). Skilled in designing efficient, flexible, and well-structured software that bridges academia with industry.
        \end{onecolentry}

	\vspace{\sectionoffset}

	\section{Experience}

	\begin{twocolentry}
		{ May 2024 – Present } \textbf{Data Science Fellow}, TMW Center for Early Learning
		+ Public Health -- Chicago, IL
	\end{twocolentry}

	\vspace{\subsectiongap}
	\begin{onecolentry}
		\begin{highlights}
			\item Raising backend code ($\sim$19k lines) coverage by 25+ percentage points
			\item Wrote code in C\#, TypeScript, JavaScript, and Python
			\item Assisted in integrating two programming teams
		\end{highlights}
	\end{onecolentry}

	\vspace{\subsectiongapd}

	\begin{twocolentry}
		{ Jan. 2020 - Feb. 2023 } \textbf{Teaching and Research Assistant}, Michigan
		State University -- East Lansing, MI
	\end{twocolentry}

	\vspace{\subsectiongap}
	\begin{onecolentry}
		\begin{highlights}
			\item Published 2 papers, where the relevant code was written in Python
			\item Assisted teaching classes, including one where we implemented SQLite
			from scratch in Python 3 \item Consistent high reviews from students
		\end{highlights}
	\end{onecolentry}

	\vspace{\subsectiongapd}

	\begin{twocolentry}
		{ Jan. 2018 - Dec. 2019 \\ \hspace{-1em}(Gap to continue at Northern) \\ May 2015 - Sep. 2016 
	\vspace{-0.7 cm}} \textbf{Product
		Development Engineer (Various Titles)}, Intel (NSG) -- Folsom, CA
	\end{twocolentry}

	\begin{onecolentry}
		\begin{highlights}
			\item Coordinated a small team of programmers (3-5 people at any given
			time) \item Helped design a testing protocol for NVMe's Power Loss Notification
			\item Influenced changes to the NVMe specification \item Rewrote internal tools
			to streamline and comply with Python 3 \item Built software models of various
			pre-market products
		\end{highlights}
	\end{onecolentry}

	\vspace{\sectionoffset}

	\section{Education}

	\begin{twocolentry}
		{ Jan. 2020 - Dec. 2022 } \textbf{Michigan State University}, Master's in Computer
		Science \& Engineering
	\end{twocolentry}

	\vspace{\subsectiongap}
	\begin{onecolentry}
		\begin{highlights}
			\item GPA: 3.85/4.0 \item {\small {\textbf{Coursework:} Discrete Logic, Distributed Systems, Foundations of Computing, Machine Learning, Graph Algorithms, Parallel Computing}}
		\end{highlights}
	\end{onecolentry}

	\vspace{\subsectiongapd}

	\begin{twocolentry}
		{ Sep. 2013 - Dec. 2018 \\ (Concurrent with Intel)\vspace{-0.35cm}} \textbf{Northern Michigan University}, BS in Computer
		Science
	\end{twocolentry}

	\vspace{\subsectiongap}
	\begin{onecolentry}
		\begin{highlights}
			\item GPA: 3.84/4.0 (Magna cum laude)
			\item {\small \textbf{Coursework:} {Algorithm Design/Analysis, Data Structures, Micro Architecture, Networking, Object-Oriented Design, Operating Systems}}
		\end{highlights}
	\end{onecolentry}

	\vspace{\sectionoffset}

	\section{Publications}

	\begin{samepage}
		\begin{twocolentry}
			{ Jan. 2022 } \textbf{Achieving Causality with Physical Clocks}
		\end{twocolentry}

		\vspace{\subsectiongap}

		\begin{twocolentry}
            {\href{https://doi.org/10.1145/3491003.3491009}{10.1145/3491003.3491009}}
			\mbox{Sandeep S Kulkarni},
			\mbox{\textbf{\textit{Olivia Appleton-Crocker}}}, \mbox{Duong Nguyen}

			\vspace{\subsectiongap}

			
		\end{twocolentry}
	\end{samepage}

	\vspace{\subsectiongapd}

	\begin{samepage}
		\begin{twocolentry}
			{ July 2020 } \textbf{Efficient Two-Layered Monitor for Partially
			Synchronous Distributed Systems}
		\end{twocolentry}

		\vspace{\subsectiongap}

		\begin{twocolentry}
            {\href{https://doi.org/10.48550/arXiv.2007.13030}{10.48550/arXiv.2007.13030}}
			\mbox{Vidhya Tekken Valapil}, \mbox{Sandeep S Kulkarni}, \mbox{Eric Torng},
			\mbox{\textbf{\textit{Olivia Appleton-Crocker}}}

			\vspace{\subsectiongap}

		\end{twocolentry}
	\end{samepage}

	\vspace{\sectionoffset}

	\section{Projects}

	\begin{twocolentry}
		{ \href{https://github.com/python/cpython}{github.com/python/cpython} } \textbf{CPython}
	\end{twocolentry}

	\begin{onecolentry}
		\begin{highlights}
			\item Added support for the UDPLite network protocol
			\item Tools Used: C, Python, Sphinx, UnitTest
		\end{highlights}
	\end{onecolentry}

	\vspace{\subsectiongapd}

	\begin{twocolentry}
		{ \href{https://euler.oliviaappleton.com}{euler.oliviaappleton.com} } \textbf{Showcase:
		Project Euler Solutions}
	\end{twocolentry}

	\begin{onecolentry}
		\begin{highlights}
			\item Solutions in 9 different languages to various math programming puzzles, including extensive prime number toolkit
			\item Tools Used: C, C+\!+, C\#, CI/CD, Fortran, Java, JavaScript, Lua,
			Makefile, Python, Rust, Sphinx, WebAssembly
		\end{highlights}
	\end{onecolentry}

	\vspace{\subsectiongapd}

	\begin{twocolentry}
		{ \hspace{-3 cm} \href{https://github.com/LivInTheLookingGlass/overpassify}{ \mbox{github.com/LivInTheLookingGlass/overpassify}} } \textbf{Overpassify}
	\end{twocolentry}

	\begin{onecolentry}
		\begin{highlights}
			\item A transpiler that turns Python code into OpenStreetMap's OverpassQL query language
			\item Tools Used: Makefile, OpenStreetMap, OverpassQL, Python
		\end{highlights}
	\end{onecolentry}

	\vspace{\sectionoffset}

	\section{Technologies}

	\begin{onecolentry}
		\textbf{Languages:} Python, C/C+\!+, C\#, Rust, JavaScript, SQL, Java, Bash,
		Fortran, Lua, SmallTalk
	\end{onecolentry}

	\begin{onecolentry}
		\textbf{Technologies:} CI/CD, Cypress, Github Actions, Makefile, Mocha, Moq, .NET, PyTest, UnitTest
	\end{onecolentry}
\end{document}
